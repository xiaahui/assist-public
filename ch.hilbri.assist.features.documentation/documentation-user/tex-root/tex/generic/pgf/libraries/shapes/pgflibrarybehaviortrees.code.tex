% Copyright 2014 by Andreas Kl�ckner
\ProvidesFile{pgflibrarybehaviortrees.code.tex}[2014/08/06 v1.0 A behavior trees library for TikZ]

% Some packages are required
\RequirePackage{tikz}
\RequirePackage{pgfmath}
\usetikzlibrary{shapes}
\usepgflibrary{intersections}

% Define a style for task symbols
\tikzset{
  every task/.style={}
}

% Add a transient option
\tikzset{
  transient/.style={every task/.append style={dashed}}
}

% Add an active option
\tikzset{
  active/.style={font=\bfseries}
}

% Define commands to draw symbols
\def\bt@selector{
  \pgf@anchor@selector@symbol
  \node[draw,anchor=center,
        every task,
        circle,
        inner sep        = 0em,
        minimum height   = 1.0em] at (\the\pgf@x,\the\pgf@y) {?};
}
\def\bt@sequence{
  \pgf@anchor@sequence@symbol
  \node[draw,anchor=center,
        every task,
        rectangle,
        inner ysep        = 0em,
        minimum height   = 1.0em] at (\the\pgf@x,\the\pgf@y) {$\rightarrow$};
}
\expandafter\def\csname bt@sequence*\endcsname{
  \csname pgf@anchor@sequence*@symbol\endcsname
  \node[draw,anchor=center,
        every task,
        rectangle,
        inner ysep        = 0em,
        minimum height   = 1.0em] at (\the\pgf@x,\the\pgf@y) {$\rightarrow$\small$^*$};
}
\expandafter\def\csname bt@selector*\endcsname{
  \csname pgf@anchor@selector*@symbol\endcsname
  \node[draw,anchor=center,
        every task,
        circle,
        inner sep        = 0em,
        minimum height   = 1.0em] at (\the\pgf@x,\the\pgf@y) {?\small$^*$};
}
\def\bt@action{
  \pgf@anchor@action@symbol
  \node[draw,anchor=center,
        every task,
        trapezium,rotate =180,
        trapezium angle  = 60,
        inner xsep       = 0em,
        minimum height   = 1.0em,
        text width       = 0.7em] at (\the\pgf@x,\the\pgf@y) {};
  %\pgf@anchor@action@symbol
  %\node[draw,anchor=center,
        %every task,
        %rectangle,rotate =  0,
        %inner xsep       = 0em,
        %minimum height   = 1.0em,
        %text width       = 2.0em] at (\the\pgf@x,\the\pgf@y) {};
}
\def\bt@condition{
  \pgf@anchor@condition@symbol
  \node[draw,anchor=center,
        every task,
        trapezium,rotate =  0,
        trapezium angle  = 60,
        inner xsep       = 0em,
        minimum height   = 1.0em,
        text width       = 0.7em] at (\the\pgf@x,\the\pgf@y) {};
  %\pgf@anchor@condition@symbol
  %\node[draw,anchor=center,
        %every task,
        %rectangle,rotate =  0,
        %inner xsep       = 0em,
        %minimum height   = 1.0em,
        %text width       = 2.0em] at (\the\pgf@x,\the\pgf@y) {};
}
\def\bt@decorator{
  \pgf@anchor@condition@symbol
  \node[draw,anchor=center,
        every task,
        diamond,
        inner xsep       = 0em,
        minimum height   = 1.0em,
        text width       = 0.7em] at (\the\pgf@x,\the\pgf@y) {};
}
\expandafter\def\csname bt@parallel sequence\endcsname{
  \csname pgf@anchor@parallel sequence@symbol\endcsname
  \node[draw,anchor=center,
        every task,
        rectangle, double,
        inner ysep       = 0em,
        minimum height   = 0.85em] at (\the\pgf@x,\the\pgf@y) {$\rightarrow$};
}
\expandafter\def\csname bt@parallel selector\endcsname{
  \csname pgf@anchor@parallel selector@symbol\endcsname
  \node[draw,anchor=center,
        every task,
        circle, double,
        inner sep        = 0em,
        minimum height   = 1.0em] at (\the\pgf@x,\the\pgf@y) {?};
}
\def\bt@semaphore{
  \bt@decorator
  \pgf@anchor@semaphore@symbol
  \pgf@xa=\pgf@x \pgf@ya=\pgf@y
  \node[draw,anchor=center,
        circle,
        inner sep        = 0em,
        minimum height   = 0.3em,
        text width       = 0.0em] at (\the\pgf@xa,\the\pgf@ya) {};
  \node[draw,anchor=center,
        circle, yshift   = 0.3em,
        inner sep        = 0em,
        minimum height   = 0.3em,
        text width       = 0.0em] at (\the\pgf@xa,\the\pgf@ya) {};
  \node[draw,anchor=center,
        circle, yshift   =-0.3em,
        inner sep        = 0em,
        minimum height   = 0.3em,
        text width       = 0.0em] at (\the\pgf@xa,\the\pgf@ya) {};
}
\def\bt@latch{
  \bt@decorator
  \pgf@anchor@latch@symbol
  \pgf@xa=\pgf@x \pgf@ya=\pgf@y
  \node[anchor=center, circle,
        minimum height   = 0.3em] at (\the\pgf@xa,\the\pgf@ya) {\textasteriskcentered};
}
\def\bt@reset{
  \bt@decorator
  \pgf@anchor@reset@symbol
  \pgf@xa=\pgf@x \pgf@ya=\pgf@y
  \node[anchor=center, circle, draw, inner sep = 0,
        minimum height   = 0.3em] at (\the\pgf@xa,\the\pgf@ya) {\tiny /};
}
\expandafter\def\csname bt@left out\endcsname{
  \csname pgf@anchor@left out@symbol\endcsname
  \pgf@xa=\pgf@x \pgf@ya=\pgf@y
  \node[draw,anchor=center,
        circle,
        inner sep        = 0em,
        minimum height   = 0.2em,
        text width       = 0.0em] at (\the\pgf@xa,\the\pgf@ya) {};
  \node[draw,anchor=center,
        circle, xshift   = 0.4em,
        inner sep        = 0em,
        minimum height   = 0.2em,
        text width       = 0.0em] at (\the\pgf@xa,\the\pgf@ya) {};
  \node[draw,anchor=center,
        circle, xshift   =-0.4em,
        inner sep        = 0em,
        minimum height   = 0.2em,
        text width       = 0.0em] at (\the\pgf@xa,\the\pgf@ya) {};
}

% Define command to import an anchor with a new name
\newcommand*\importanchoras[3]{%
  \expandafter\gdef\csname pgf@anchor@\pgf@sm@shape@name @#3\expandafter\endcsname
                   \expandafter{\csname pgf@anchor@#1@#2\endcsname}}

% Define a command to define common stuff
\def\bt@common{

  % Inherit save anchors from rectangle for the text box
  \inheritsavedanchors[from=rectangle] %   this is nearly a rectangle
  
  % Directly inherit some anchors for the text box
  \inheritanchor[from=rectangle]{text}
  \inheritanchor[from=rectangle]{base}
  \inheritanchor[from=rectangle]{base east}
  \inheritanchor[from=rectangle]{base west}
  \inheritanchor[from=rectangle]{south}
  \inheritanchor[from=rectangle]{south east}
  \inheritanchor[from=rectangle]{south west}
  \inheritanchor[from=rectangle]{mid}
  \inheritanchor[from=rectangle]{mid east}
  \inheritanchor[from=rectangle]{mid west}
  
  % Inherit some renamed anchors for the label
  \importanchoras{rectangle}{east}      {label east}
  \importanchoras{rectangle}{west}      {label west}
  \importanchoras{rectangle}{center}    {label center}
  \importanchoras{rectangle}{north}     {label north}
  \importanchoras{rectangle}{north east}{label north east}
  \importanchoras{rectangle}{north west}{label north west}
  
  % For now inherit also center
  %\inheritanchor[from=rectangle]{center}
  \anchor{center}{
    \expandafter\csname pgf@anchor@\shapename @symbol\endcsname}
  
  % Remember who we are
  \savedmacro{\shapename}{
    \edef\shapename{\pgf@sm@shape@name}}
    
  % Define a size for the symbol
  \saveddimen{\symbolsize}{
    \setlength{\pgf@x}{1em}}
    
  % Remember the label width
  \saveddimen{\labelwidth}{
    \pgf@x=\wd\pgfnodeparttextbox
  }
  
  % Define the symbol anchor
  \anchor{symbol}{
    \expandafter\csname pgf@anchor@\shapename @base\endcsname
    \setlength{\pgf@yc}{2.5ex}       \advance\pgf@y by 1.0\pgf@yc
    \setlength{\pgf@yc}{\symbolsize} \advance\pgf@y by 0.5\pgf@yc
  }
  
  % Define the north anchor
  \anchor{north}{
    \expandafter\csname pgf@anchor@\shapename @base\endcsname
    \setlength{\pgf@yc}{2.5ex}       \advance\pgf@y by 1.0\pgf@yc
    \setlength{\pgf@yc}{\symbolsize} \advance\pgf@y by 1.0\pgf@yc
  }
  
  % Define the west anchor to lie on order
  \anchor{west}{
    \expandafter\csname pgf@anchor@\shapename @center\endcsname
    \setlength{\pgf@yc}{-2cm}        \advance\pgf@x by 1.0\pgf@yc
    \expandafter\csname pgf@anchor@\shapename @border\endcsname
    }

  % Define the east anchor to lie on border
  \anchor{east}{
    \expandafter\csname pgf@anchor@\shapename @center\endcsname
    \setlength{\pgf@yc}{+2cm}        \advance\pgf@x by 1.0\pgf@yc
    \expandafter\csname pgf@anchor@\shapename @border\endcsname
    }

  % The border will either be on the text or in the top node
  \anchorborder{
    % Save the requested point!
    \pgf@xb =\pgf@x \pgf@yb =\pgf@y 
    %
    % Save the \northeast corner
    \northeast
    \pgf@xa =\pgf@x  \pgf@ya =\pgf@y 
    %
    % Check, if we have a label and if we are below \northeast
    \newif\iflabel\labelfalse
    \ifdim\labelwidth>0pt
      \ifdim\pgf@yb<\pgf@ya
        \labeltrue
      \fi
    \fi
    \iflabel
    %\ifdim0pt<\pgf@ya % Always use the bigger bounding box
    %
    % This needs the rectangle code
      \southwest                                                   % southwest corner
      \pgf@xa =\pgf@x \pgf@ya =\pgf@y                              % ... save it to A => lower left corner
      \northeast                                                   % northeast corner
      \advance \pgf@x  by-\pgf@xa \advance \pgf@y  by-\pgf@ya      % ... add the southwest distance
                                  \advance \pgf@y  by\symbolsize   % ... add the symbol size !!
      \pgf@xc =.5\pgf@x \pgf@yc =.5\pgf@y                          % ... save half of it to C => half diameter of rectangle
      \advance \pgf@xa by \pgf@xc \advance \pgf@ya by \pgf@yc      % add half rectangle size to A => center of rectangle
      %\advance \pgf@xb by-\pgf@xa \advance \pgf@yb by-\pgf@ya      % move requested point relative to center of rectangle
      \edef \pgf@marshal {                                         % define an intersection command
        \noexpand \pgfpointborderrectangle                         % ... using a rectangular border
         {\noexpand \pgfqpoint {\the \pgf@xb }{\the \pgf@yb }}     % ... the guess point
         {\noexpand \pgfqpoint {\the \pgf@xc }{\the \pgf@yc }}}    % ... half the rectangle diagonal
      \pgf@process {\pgf@marshal }                                 % execute it
      \advance \pgf@x by\pgf@xa \advance \pgf@y by\pgf@ya          % move requested point relative to center of rectangle
    %
    % If we are above, just return some anchor
    \else
    % This needs the rectangle code
      \csname pgf@anchor@\shapename @symbol\endcsname              % the symbol center
      \pgf@xa =\pgf@x \pgf@ya =\pgf@y                              % ... save it to A => center of rectangle
      \setlength{\pgf@x}{\symbolsize}                              % the symbol size
      \pgf@xc = 1.0\pgf@x \pgf@yc = 0.5\pgf@x                      % ... save half symbol square to C => half diameter of rectangle
      %\advance \pgf@xb by-\pgf@xa \advance \pgf@yb by-\pgf@ya      % move requested point relative to center of rectangle
      \edef \pgf@marshal {                                         % define an intersection command
        \noexpand \pgfpointborderrectangle                         % ... using a rectangular border
         {\noexpand \pgfqpoint {\the \pgf@xb }{\the \pgf@yb }}     % ... the guess point
         {\noexpand \pgfqpoint {\the \pgf@xc }{\the \pgf@yc }}}    % ... half the rectangle diagonal
      \pgf@process {\pgf@marshal }                                 % execute it
      \advance \pgf@x by\pgf@xa \advance \pgf@y by\pgf@ya          % move requested point relative to center of rectangle
    \fi
  }
  
  % Define the background path
  \backgroundpath{
    \pgfpathrectanglecorners{\southwest}{\northeast}
    {\csname bt@\shapename\endcsname}
  }
  
}

% Define the shapes!
\pgfdeclareshape{selector} {\bt@common}
\pgfdeclareshape{sequence} {\bt@common}
\pgfdeclareshape{action}   {\bt@common}
\pgfdeclareshape{condition}{\bt@common}
\pgfdeclareshape{decorator}{\bt@common}
\pgfdeclareshape{parallel sequence}{\bt@common}
\pgfdeclareshape{parallel selector}{\bt@common}
\pgfdeclareshape{semaphore}{\bt@common}
\pgfdeclareshape{latch}    {\bt@common}
\pgfdeclareshape{reset}    {\bt@common}
\pgfdeclareshape{sequence*}{\bt@common}
\pgfdeclareshape{selector*}{\bt@common}
\pgfdeclareshape{left out} {\bt@common}

