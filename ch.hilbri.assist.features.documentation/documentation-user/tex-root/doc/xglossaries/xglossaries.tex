\documentclass{scrartcl}

% Some default packages
\usepackage[utf8]{inputenc}
\usepackage[T1]{fontenc}
\usepackage{lmodern}
\usepackage{microtype}
\usepackage[ngerman, english]{babel}
\usepackage[colorlinks=true]{hyperref}


% Some packages for this doc
\usepackage{xcolor}
\usepackage{listings}
\lstset{%
	language=[LaTeX]{TeX},
	basicstyle=\small\ttfamily,
	commentstyle=\color{black!80!white},
	showstringspaces=false,
	keywordstyle=\color{blue}\bfseries,
	morekeywords={newacronym, glosmanref, ac, newsym, sym, Sym, newsymcat, newsymincat, defaultsymcat}}



%
% Title etc
%%%%%%%%%%%%%%%%%%%%%%%%%%%%%%%%%%%%%%%%%%%%%%%%%%%%%%%%%%%%%%%%%%%%%
\title{xGlossaries Package}
\subtitle{An extension to the glossaries packages}
\author{Andreas Knoblach}

%
% Load xglossaries and hack it
%%%%%%%%%%%%%%%%%%%%%%%%%%%%%%%%%%%%%%%%%%%%%%%%%%%%%%%%%%%%%%%%%%%%%
\usepackage[nomain,shortcuts]{xglossaries}

%
% Define glossaries
%%%%%%%%%%%%%%%%%%%%%%%%%%%%%%%%%%%%%%%%%%%%%%%%%%%%%%%%%%%%%%%%%%%%%
\newglossary[alg]{acronym}{acr}{acn}{Abbreviations}
\glosmanref{acronym}

\newglossary[slg]{symbol}{syr}{syn}{Symbols}
\glosmanref{symbol}

\makeglossaries


%
% Define glossaries entries
%%%%%%%%%%%%%%%%%%%%%%%%%%%%%%%%%%%%%%%%%%%%%%%%%%%%%%%%%%%%%%%%%%%%%
\newacronym[manref, pluralalias=TLAs]{TLA}{TLA}{three letter acronym}

\newsymcat{aero}{Aerodynamics}
\newsymcat{cont}{Control Theory}

\defaultsymcat{aero}
	\newsymincat[unit={m/s}]{Qhh}{Q_{hh}}{generalized aero}
	\newsymincat{Qjj}{Q_{jj}}{aero}\symshortcut{Qjj}

\defaultsymcat{cont}
		\newsymincat{A}{A}{state matrix}\symshortcut[Amat]{A}
		\newsymincat{B}{B}{input matrix}
		\newsymincat{C}{C}{output matrix}

%
% Document
%%%%%%%%%%%%%%%%%%%%%%%%%%%%%%%%%%%%%%%%%%%%%%%%%%%%%%%%%%%%%%%%%%%%%
\begin{document}
\glsaddall

%
% Title
%
\maketitle

%
% Introduction
%
\section{Introduction}
The \texttt{xgloassries} package provides some useful extensions to the \texttt{glossaries} packages. This documentation explains the additional features only. The basic setup and use of glossaries is explained in the \texttt{glossaries} documentation. It is recommended to read that documentation, too.

%
% Package Loading and Options
%
\section{Package Loading and Options}
To load the package, use
\begin{lstlisting}
\usepackage[Key1=Value1, ...]{xglossaries}
\end{lstlisting}
There are three aditonal keys, which are explained below. All other keys are forwared to the \texttt{glossaries} packages.

%
% Manual Referencing
%
\section{Manual Referencing}
This featurea allows to manually control, to which pages are refered in the gloassary. To deactivate this feature use
\begin{lstlisting}
\usepackage[manref=false]{xglossaries}
\end{lstlisting}

In order to create a gloassry entry, whose referecenc list (in the gloassary packge also refered to as number list), uses the \texttt{manref} key for \texttt{\textbackslash newglossaryentry}, \texttt{\textbackslash newacronym} etc. commands,  e.\,g.
\begin{lstlisting}
\newacronym[manref]{TLA}{TLA}{three letter acronym}
\end{lstlisting}
To activate the manual referencing for all entries of one glossary, use the \texttt{\textbackslash glosmanref}
\begin{lstlisting}
\glosmanref{acronym}
\end{lstlisting}
This command has to be called, before entries are defined.

You can use any entry as explained in \texttt{glossaries} documentation using the \texttt{\textbackslash gls} and \texttt{\textbackslash ac} commands. To indicate that a usage shall be referred in the glossaries, use the \texttt{ref} key, e.\,g.
\begin{lstlisting}
\ac[ref]{TLA}
\end{lstlisting}
for \ac[ref]{TLA}.

%
% Symbol Tools
%
\section{Plural Alias}
This feature allows to define an alias for the plural of an acronym. To deactivate this feature use
\begin{lstlisting}
\usepackage[PluralAlias=false]{xglossaries}
\end{lstlisting}

The \texttt{glossaries} package provides several commands to create the plural of an acronym, e.\,g. \lstinline!\acp{TLA}! for \acp{TLA}. While this works well in principle, the usage correponds not to the natural language usage. A more inuitive interface would be \lstinline!\acp{TLAs}!. This can be achived using \lstinline!pluralalias! key, e.\,g.:
\begin{lstlisting}
\newacronym[manref, pluralalias=TLAs]%
	{TLA}{TLA}{three letter acronym}
\end{lstlisting}
Now, \lstinline!\ac{TLAs}! results in \ac{TLAs}.


%
% Symbol Tools
%
\section{Symbol Tools}
This feature provides some tools for creating a nomenclature using the glossaries package. To deactivate this feature use
\begin{lstlisting}
\usepackage[SymbolTools=false]{xglossaries}
\end{lstlisting}

A command to add new symbols to a glossary \texttt{\textbackslash newsym} is provided:
\begin{lstlisting}
\newsym[unit={m/s}]{Qhh}{Q_{hh}}{generalized aero}
\end{lstlisting}
The additional key \texttt{unit} is an alias for for the fields \texttt{user1}/\texttt{useri} of a glossary. To refer to a symbol, you can use the command \texttt{\textbackslash sym} and \texttt{\textbackslash Sym}, e.\,g.
\begin{lstlisting}
\sym[ref]{Qhh}
\end{lstlisting}
for \sym[ref]{Qhh}.

Further more, there you can use \texttt{\textbackslash newsymcat} to create a category for symbols, e.\,g.
\begin{lstlisting}
\newsymcat{aero}{Aerodynamics}
\end{lstlisting}
To add a symbol to a category use the \texttt{parent} key. Alternatively, use \texttt{\textbackslash newsymincat} to add a symbol to a default category. The default category can be chosen (or changed) by \texttt{\textbackslash defaultsymcat}. This is a complete example:
\begin{lstlisting}
% Create categroies
\newsymcat{aero}{Aerodynamics}
\newsymcat{cont}{Control Theory}

% Add smbols to aero
\defaultsymcat{aero}
	\newsymincat[unit={m/s}]{Qhh}{Q_{hh}}{generalized aero}
	\newsymincat{Qjj}{Q_{jj}}{aero}

% Add smbols to cont
\defaultsymcat{cont}
	\newsymincat{A}{A}{state matrix}
	\newsymincat{B}{B}{input matrix}
	\newsymincat{C}{C}{output matrix}
\end{lstlisting}
In order to print that glossary, it is highly recommended to use the \texttt{longSymbolCats} glossary style, see next section.

% Create shortcurt
Finally, you can use \texttt{\textbackslash symshortcut} to creat shortcuts for symbols.
\begin{lstlisting}
	\symshortcut{Qjj}%      % Creates \Qjj[#1]=\sym*[#1]{Qjj}
	\symshortcut[Amat]{A}%  % Creates \Amat[#1] =\sym*[#1]{A}
\end{lstlisting}
Additionally a command \texttt{\textbackslash Qjj?} and \texttt{\textbackslash Amat?} is created, which call the non-starred version of \texttt{\textbackslash sym}.

Some tests: \Qjj \Qjj?[ref] \Amat[ref] \Amat?

%
% Glossary Styles
%
\section{Glossary Styles}
Finally, this package provides some additional glossary styles.  To deactivate this feature use
\begin{lstlisting}
\usepackage[DefineStyles=false]{xglossaries}
\end{lstlisting}

For acronyms, the styles \texttt{longAcronym} and \texttt{longAcronymHeader} are provided:

\bigskip
\printglossary[type=acronym, style=longAcronymHeader]

For symbols, the styles \texttt{longSymbol} and \texttt{longSymbolHeader} are provided. Additionally, the style \texttt{longSymbolCatsHeader} in combination with the category tools of the preceding section is provided:

\bigskip
\printglossary[type=symbol, style=longSymbolCatsHeader]


\end{document}


