% Document class
\documentclass[figures=dlr]{dlrbook}
%\PassOptionsToPackage[institute=FA-VPT, site=STD]{dlrinstitutes} % Loads instute and site information.
%\dlrsite{STD}                                                    % Loads a different site than the default.
%                                                                 % See also the title and author definitions below.
%\usepackage{dlrlayout}

% Load packages for encoding and languages
\usepackage[latin1]{inputenc}
\usepackage[T1]{fontenc}
\usepackage[ngerman,english]{babel}
\usepackage{dlrsecondpage}

% Subfigures
\usepackage{caption}
\usepackage{subcaption}

% Load package for tables
\usepackage{booktabs, array}
\newcolumntype{C}[1]{>{\centering}m{#1}}

% Load hyperref
\usepackage[ pdfauthor  = {Andreas Kl�ckner},
                                                 pdftitle   = {The DLR report layout},
                                                 colorlinks = true, 
                                                 linkcolor  = black,  
                                                 citecolor  = black,
                                                 urlcolor   = blue,
                                                 pdfstartview = {Fit}
                                        ]{hyperref}
                                        
% Change the page layout
%\KOMAoptions{DIV=9, headinclude=false, footinclude=false, headheight=2.2cm} %bcor=10mm
%\KOMAoptions{headinclude=false, footinclude=false, headheight=2.2cm} %bcor=10mm
%\recalctypearea
%\areaset[0cm]{}
                                        
% Define title and author
\subject     {RM Latex}
\title       {DLR report layout}
\subtitle    {An easy to use DLR layout package}
\author      {Andreas Kl�ckner \& Andreas Knoblach}
%\publishers {Institute for Robotics and Mechatronics   \\
%             Department of System Dynamics and Control }
\dlrinstitute{RMC-SR}
%\dlrsite     {OP} % You may load a different site than the default...
\telnumber   {3305}
\faxnumber   {1441}
\email       {andreas.kloeckner@dlr.de}
\reportno    {IB-XY unbekannt}
\date        {\today}
\addversion  {1.1}{27.08.2012}{Changed logos}{kloe\_ad}


% Begin document
\begin{document}

%\pagenumbering{roman}
\titlematter
\settitle{type=plain}
\maketitle
\secondpage

\frontmatter
\tableofcontents

\mainmatter


%%%%%%%%%%%%%%%%%%%%%%%%%%%%%%%%%%%%
% The Package                      %
%%%%%%%%%%%%%%%%%%%%%%%%%%%%%%%%%%%%
\chapter{Introduction}
This class provides a DLR layout scheme. It allows an easy change of the title page and of the page layout. In addition, it is possible to use the DLR corporate identity fonts with a suitable math font: \texttt{cmbright}.
\begin{equation}
E = m c^2
\label{eq:example}
\end{equation}

However, the package bases on the established KOMA-Skript and tries to keep as much features as possible of it "alive".

%%%%%%%%%%%%%%%%%%%%%%%%%%%%%%%%%%%%
% Package Usage                    %
%%%%%%%%%%%%%%%%%%%%%%%%%%%%%%%%%%%%
\chapter{Usage}

\section{The \texttt{dlrtitle} Package}
This package renews \texttt{\textbackslash maketitle} command in order to change the appearance of the title page using the capabilities of TikZ. In addition, the command \texttt{\textbackslash reportno} is provided, which is similar to the \texttt{\textbackslash author} command.

The layout of the title page can be set up using the \texttt{\textbackslash settitle} command. It accepts key/value pairs. You can set the \texttt{type=plain} to chose a simple report layout. Setting \texttt{type=areas} creates a print category layout with the research areas' corporate design pictures. The research areas can be set with \texttt{area=aeronautics}, \texttt{area=space}, \texttt{area= transport}, \texttt{area=energy} and \texttt{area=custom}.
If \texttt{area=custom}, the user must supply the image \texttt{custom.*} in an accepted file format. An aspect ratio of 15/9 is advised, but not needed.
Acceptable formats are e.g. \texttt{custom.eps}, if you are using ps\LaTeX, or \texttt{custom.png} for pdf\LaTeX.

\section{The \texttt{dlrheaders} Package}
This package modifies the head and the foot line. Especial a DLR water mark and the report number is printed in the foot line. If the command \texttt{\textbackslash reportno} is not yet defined it is created. This package uses the capabilities of the \texttt{scrpage2} package.

\section{The \texttt{dlrelements} Package}
This package modifies the document fonts for titles and captions. It additionally changes the layout of tables and figures. Use \texttt{\textbackslash tablehead} to format table headings. See figure \ref{elements:figure} and table \ref{elements:table} for an example.

\begin{figure}[t]%
  \begin{subfigure}[b]{3cm}\centering
    \dlradler
    \caption{DLR adler}
    \label{elements:figure:adler1}
  \end{subfigure}
  \begin{subfigure}[b]{8cm}\centering
    \dlrlogo
    \caption{DLR logo}
    \label{elements:figure:logo1}
  \end{subfigure}
  \caption{Figure typeset with DLR elements}
  \label{elements:figure}
\end{figure}

\begin{table}\centering%
\begin{tabular}{lcr}
\dlrtablehead
 DLR   & layout & test   \\
 looks & nice,  & right? \\
 a     & second & row
\\\bottomrule
\end{tabular}
\caption{Table typeset with DLR elements}
\label{elements:table}
\end{table}

The package provides the option \texttt{figures}, which takes the values \texttt{plain} or \texttt{dlr}. If \texttt{dlr} is set, the figure appear as seen in figure \ref{elements:figure}. Otherwise, the regular figure appearance is provided. The package options can be provided to the package directly or to the dlr document classes.


\section{The \texttt{dlrreprt} Class}
The packages above are combined within the \texttt{\textbackslash dlrreprt} class, which can be loaded with \texttt{\textbackslash documentclass\{dlrreprt\}} command instead of the \texttt{scrreprt} class. This will load the class with the default options.

The class is based on the \texttt{scrreprt} class. Options not recognised by the \texttt{dlrreprt} class are passed on to \texttt{scrreprt}.

\subsection{Option \texttt{frutiger}}
By default, the package also loads the DLR corporate identity fonts provided by the package \texttt{frutiger} with the \texttt{family} option. In order to provide a consistent layout with formulae, the package \texttt{cmbright} is loaded for math fonts. In oder to avoid changing default fonts, you can use \texttt{\textbackslash documentclass[frutiger=false]\{dlrreprt\}}.

\subsection{Option \texttt{title}}
By default, the package changes the appearance of the title page by renewing the command \texttt{\textbackslash maketitle}. Use \texttt{\textbackslash documentclass[title=false]\{dlrreprt\}}, if you wish to use the default title page.

\subsection{Option \texttt{page}}
By default, the package changes the page layout. This includes a header with the DLR logo, the publisher (usually the institute) and the current chapter. The footer is filled with either the document title or the report number and a page numbering, which also states the total count of pages in the body. If you wish to preserve the default page layout, use \texttt{\textbackslash documentclass[page=false]\{dlrreprt\}}.

\subsection{Option \texttt{elements}}
By default, the \texttt{dlrelements} package is loaded. You can prevent that by specifying \texttt{\textbackslash documentclass[elements=false]\{dlrreprt\}}.

\subsection{Command \texttt{\textbackslash reportno}}
The package provides a new command \texttt{\textbackslash reportno}, which similar to the \texttt{\textbackslash author} command in that it creates a variable for use in the title page. This command should be used to insert the internal report number of the document.

\clearpage
\section{The \texttt{dlrbook} Class}
The \texttt{\textbackslash dlrbook} class is similar to the \texttt{\textbackslash dlrreprt} class and accepts the same options. Instead of the \texttt{scrreprt} class is bases on the \texttt{scrbook} class. The main advantages of this \texttt{scrbook}  class is, that it is optimised for a two-sided book layout. In addition the commands \texttt{\textbackslash frontmatter}, \texttt{\textbackslash mainmatter}  and \texttt{\textbackslash backmatter} are provide to change among others the page numbering within the front matter etc. For further information it is referred to the documentation of KOMA-Script.

\subsection{Command \texttt{\textbackslash titlematter}}
In addition to the \texttt{\textbackslash frontmatter} command \texttt{\textbackslash titlematter} can be used to change the page numbering of the title page.

\section{The \texttt{dlrsecondpage} Package}
This package provides a second page containing an identification sheet for the document and a version history. The following commands can be used to fill out information in the identification sheet:
\begin{table}[bh]
        \centering
        \caption{Document identification provided by the \texttt{\textbackslash dlrsecondpage} package}
        \begin{tabular}{ll}
                \texttt{\textbackslash title}        & The document title \tabularnewline
                \texttt{\textbackslash subject}      & The general subject\tabularnewline
                \texttt{\textbackslash keywords}     & Some keywords\tabularnewline
                \texttt{\textbackslash restrictions} & Access restrictions\tabularnewline
                \texttt{\textbackslash author}       & The author\tabularnewline
                \texttt{\textbackslash telnumber}    & The author's internal telephone number\tabularnewline
                \texttt{\textbackslash faxnumber}    & The author's internal fax number\tabularnewline
                \texttt{\textbackslash email}        & The author's email\tabularnewline
                \texttt{\textbackslash savedby}      & The the last one to save the file\tabularnewline
                \texttt{\textbackslash coauthor}     & Contributors to the document\tabularnewline
                \texttt{\textbackslash approvedby}   & Who checked the document?\tabularnewline
                \texttt{\textbackslash releasedby}   & Who released the document to the access-granted group? \tabularnewline
        \end{tabular}
        \label{tab:DLRsecondpage}
\end{table}

The packages relies on the \texttt{listofversions} package for version handling. It additionally uses the \texttt{dlrinstitutes} package to retrieve information on the institute. It also uses the \texttt{username} package to set defaults for \texttt{\textbackslash savedby}. The second page can be generated using \texttt{\textbackslash secondpage}.


%%%%%%%%%%%%%%%%%%%%%%%%%%%%%%%%%%%%
% Installation                     %
%%%%%%%%%%%%%%%%%%%%%%%%%%%%%%%%%%%%
\chapter{Installation}
\begin{enumerate}
        \item Copy the package to any desired path. It is important that you do not change the directory structure.
        \item Open the MikTeX Settings GUI and add the package path to the MikTeX root directory list.
        \item You may need to refresh the filename database (FNDB).
        \item Note: This package collection requires the \texttt{frutiger} package to work correctly. How you can install this package is explicitly explain in the corresponding documentation.
\end{enumerate}


\end{document}



% eof

%%% Local Variables:
%%% mode: latex
%%% TeX-master: t
%%% End:
