% Document class
\documentclass[]{article}

% Load packages for encoding and languages
\usepackage[latin1]{inputenc}
\usepackage[T1]{fontenc}
\usepackage[english]{babel}


% Load package for tables
\usepackage{booktabs, array}
\newcolumntype{C}[1]{>{\centering}m{#1}}


% Load dlrlogos
\usepackage{dlrlogos}

					
% Define title and author
\title{dlrlogos.sty}
\author{Andreas Knoblach}
\date      {\today}

% Begin document
\begin{document}
\maketitle

This packages provides several DLR logos drawn with TikZ. The possible logos are compiled in table \ref{tab:DLRlogos}. There are command to draw the logos in normal text as well as commands for drawing the logos within a TikZ-environment. The latter one accept an optional argument, i.\,e. a coordinate where the logo is to be drawn. The \texttt{\textbackslash dlrlogo} command recognizes, if the current language selected via the \texttt{babel} package is not (n)german, and adds the english name of the DLR to the logo.
\begin{table}[bh]
	\centering
	\caption{DLR logos provided by the \texttt{\textbackslash dlrlogos} Package}
	\begin{tabular}{m{7em}m{9em}C{18.5em}}
		\toprule
		\textbf{Command}                       & \textbf{Command in TikZ}                 &\textbf{Logo}              \tabularnewline
%		\texttt{\textbackslash minidlradler}   & \texttt{\textbackslash tikzminidlradler} & use \minidlradler   inline \tabularnewline\addlinespace
		\texttt{\textbackslash dlradler}       & \texttt{\textbackslash tikzdlradler}     & use \dlradler~      separately \tabularnewline\addlinespace
		\texttt{\textbackslash dlradlerinline} & not available                            & use \dlradlerinline inline     \tabularnewline\addlinespace
		\texttt{\textbackslash dlrlogo}        & \texttt{\textbackslash tikzdlrlogo}     & \dlrlogo                       \tabularnewline\bottomrule
	\end{tabular}
	\label{tab:DLRlogos}
\end{table}


\end{document}