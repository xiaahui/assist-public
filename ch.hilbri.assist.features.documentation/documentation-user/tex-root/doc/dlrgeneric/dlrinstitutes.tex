% Document class
\documentclass[]{article}

% Load packages for encoding and languages
\usepackage[latin1]{inputenc}
\usepackage[T1]{fontenc}
\usepackage[ngerman, english]{babel}

					
% Define title and author
\title{dlrinstitutes.sty}
\author{Andreas Kl�ckner}
\date      {\today}

% Begin document
\begin{document}
\maketitle

%%%%%%%%%%%%%%%%%%%%%%%%%%%%%%%%%%%%
% The Package                      %
%%%%%%%%%%%%%%%%%%%%%%%%%%%%%%%%%%%%
\section{The package}
The package \texttt{dlrinstitutes} provides DLR institute and site information to other packages via simple interface commands. The package can be loaded by \texttt{\textbackslash usepackage\{dlrinstitutes\}}. This will provide the following commands to the user:

\subsection{\textbackslash dlrinstitute\{<abbreviation>\}}
Use this command to load the given institute's information. The information must be provided in a file called \texttt{dlrinstitutes\_inst<abbreviation>.sty}. if no such file is present, an error will be issued. Default is \textbackslash dlrinstitute\{RMC\}. Please see the respective institute definition file for an example, how to define you own institute. You can also specify the institute, by passing the option "institute=<abbreviation>" to the package.

\subsection{\textbackslash dlrsite\{<abbreviation>\}}
This command is the equivalent for site information. Default is \textbackslash dlrsite\{OP\}. You can also call this command from within your institute definition. You can also specify the site, by passing the option "site=<abbreviation>" to the package.

\subsection{Order of loading and using definitions}
In order to use data of institutes or site definitions, these definitions must be loaded prior to using them with one of the two commands described above.
The order of loading definitions (and thus also the arguments to the package) is important, as loading a definition makes it the default for use in the document.
Note that institutes usually define a default site. If you wish to use a different site other than the default site, you must load the site after the institute definition.
Data can only be loaded in the preamble. Make sure, you load all institutes and sites there!

\subsection{\textbackslash DLR\{<info type>\}}
This command is used to retrieve global information about the DLR. The command respects language changes by the babel package. E.g. \textbackslash DLR\{name\} will return "German Aerospace Center", when used with english language. Typical info types are listed in table \ref{tab:DLRinfo}.
\begin{table}[h]%
	\centering%
	\begin{tabular}{ll}
		abbreviation & = DLR \\
		name         & e.g. German Aerospace Center \\
		url          & e.g. www.dlr.de/en
	\end{tabular}
	\caption{\textbackslash DLR info types}
	\label{tab:DLRinfo}
\end{table}

\subsection{\textbackslash INST[<institute definition>]\{<info type>\}}
This command retrieves information about the institute. 
The requested information must be defined by the institute definition file and loaded in the preamble.
Otherwise, an error is issued. 
If the optional argument is omitted, the current default institute is used.
The default institute can be switched using \textbackslash dlrinstitute.
Typical info types are listed in table \ref{tab:INSTinfo}.
\begin{table}[h]%
	\centering%
	\begin{tabular}{ll}
		abbreviation & The abbreviation of the department \\
		name         & The name of the institute \\
		department   & The name od the department \\
		head         & The head of the department \\
		operator     & The operator's dial-through \\
		fax          & The operator's fax \\
		url          & The departments web site
	\end{tabular}
	\caption{\textbackslash INST info types}
	\label{tab:INSTinfo}
\end{table}


\subsection{\textbackslash SITE[<site definition>]\{<info type>\}}
This command retrieves information about the site. 
The requested information must be defined by the site definition file and loaded in the preamble.
Otherwise, an error is issued. 
If the optional argument is omitted, the current default site is used.
The default site can be switched using \textbackslash dlrsite.
Typical info types are listed in table \ref{tab:SITEinfo}.
\begin{table}[h]%
	\centering%
	\begin{tabular}{ll}
		abbreviation & The site's abbreviation \\
		name         & The site's full name\\
		street       & The site's street adress \\
		zip          & The site's ZIP code \\
		city         & The name of the site's city \\
		prefix       & The site telephone area code \\
		subscriber   & The site telephone subscriber number \\
		operator     & The site's operator dial-through \\
		url          & The site's web..., well site
	\end{tabular}
	\caption{\textbackslash SITE info types}
	\label{tab:SITEinfo}
\end{table}



\end{document}



% eof
