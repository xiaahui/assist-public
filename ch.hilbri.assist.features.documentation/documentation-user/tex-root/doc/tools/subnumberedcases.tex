% This is file `subnumberedcases.tex',
% it demonstrated the possibilities of  the subnumberedcases package
%
% REVISIONS:    2011-11-28 initial release   Andreas Knoblach <andreas.knoblach@dlr.de>
%
% Contact       Andreas Knoblach,  Andreas.knoblach@dlr.de
% Copyright (C) 2008-2011 DLR Robotics and Mechatronics         __/|__
%                                                              /_/_/_/  
%                                                                |/ DLR


\documentclass[12pt,fleqn,DIV=14]{scrartcl}

% Some default package
\usepackage[english]{babel}
\usepackage[latin1]{inputenc}
\usepackage[T1]{fontenc}
\usepackage{lmodern}  

% AMS and subnumberedcases package
\usepackage{amsmath}
\usepackage{subnumberedcases}

% Booktabs and hyperref
\usepackage{booktabs}
\usepackage[colorlinks]{hyperref}

% Change couting of equaions
\numberwithin{equation}{section} 

\author{Andreas Knoblach}
\title{The Subnumberedcases Package}

\begin{document}
\maketitle
\addsec{Abstract}
This not a real docu, it only demonstrates the possibilities of the subnumberedcases package.
\tableofcontents
\clearpage

\section{Test Subequations*}
\subsection{Subequations}
Normal \texttt{subequations} environment:
\begin{align}
	ASD\label{SE1}
\end{align}
\begin{subequations}
	\begin{align}
		ASD\label{SE2a} \\
		ASD\label{SE2b} \\
		ASD\label{SE2c} 
	\end{align}
\end{subequations}

\subsection{Subequations*}
The \texttt{subequations*} environment:
\begin{align}
	ASD\label{SE3}
\end{align}
\begin{subequations*}
	\begin{align}
		ASD\label{SE3a} \\
		ASD\label{SE3b} \\
		ASD\label{SE3c} 
	\end{align}
\end{subequations*}
Note that there are the equations \eqref{SE3} and \eqref{SE3a} but no (\ref{SE1}a).

\subsection{References}
\begin{center}
	\begin{tabular}{lr}
		\toprule
			\textbf{Label} & \textbf{Reference} \\ \midrule
			SE1  & \eqref{SE1}  \\ \midrule
			SE2a & \eqref{SE2a} \\
			SE2b & \eqref{SE2b} \\
			SE2c & \eqref{SE2c} \\ \midrule
			SE3  & \eqref{SE3}  \\ \midrule
			SE3a & \eqref{SE3a} \\
			SE3b & \eqref{SE3b} \\
			SE3c & \eqref{SE3c} \\ \bottomrule
	\end{tabular}
\end{center}
\clearpage

\section{Test Subnumberedcases}
\subsection{Subnumberedcases}
The \texttt{subnumberedcases} environment:
\begin{align}
	ASD\label{E1}
\end{align}
\begin{subnumberedcases}{ASD ASD ASD ASD\label{E2}}
	& ASD\label{E2a} \\
	& 1+1-1+1-1+1-1+1-1+1-1+1-1+1-1+1-1+1-1 \label{E2b} \\
	& ASD\label{E2c}
\end{subnumberedcases}
\begin{align}
	ASD  \label{E3}
\end{align}
	
\subsection{Indent of Subnumberedcases}
The \texttt{subnumberedcases[maxindent]} environment\footnote{The default \texttt{maxindent} is 1cm. It can be changed by \texttt{\textbackslash\textbackslash setlength\{\textbackslash\textbackslash SNCmaxindent\}\{1cm\}}.}:
\begin{subnumberedcases}[2cm]{\text{Blablabbla } \frac{\frac{1}{2}}{\frac{1}{2}} =\label{E4}}
	& 1\label{E4a} \\
	& \frac{1}{2}\label{E4b}
\end{subnumberedcases}

\subsection{References}
\begin{center}
	\begin{tabular}{lr}
		\toprule
			\textbf{Label} & \textbf{Reference} \\ \midrule
			E1  & \eqref{E1}  \\ \midrule
			E2  & \eqref{E2}  \\
			E2a & \eqref{E2a} \\
			E2b & \eqref{E2b} \\
			E2c & \eqref{E2c} \\ \midrule
			E3  & \eqref{E3}  \\ \midrule
			E4a & \eqref{E4a} \\
			E4b & \eqref{E4b} \\ \bottomrule
	\end{tabular}
\end{center}

\section{Problems, Warnings and Errors}
The subnumberedcases package requires that align environment flushes equations left. To achieve this select the option \texttt{fleqn} of the used document class or of amsmath.


\end{document}

