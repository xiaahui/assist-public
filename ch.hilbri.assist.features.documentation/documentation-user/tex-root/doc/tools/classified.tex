\documentclass{article}

% Take care of the input encoding
\usepackage[ansinew]{inputenc}
\usepackage[T1]{fontenc}

% Manage revisions
\usepackage{listofversions}
\addversion{0.1}{2012-04-27}{Initial version of the document}{kloe\_ad}

% Use classified package
\usepackage{xcolor}
\usepackage{classified}

% Define class one classification scheme
\classificationclass{classone}{
  one=\classifiedremark{\colorbox{white}{
                        Level One Classification}}, 
  two=\classifiedremark{\colorbox{red}  {
                        Classification Level Two}}}
                        
% Define class two classification scheme
\classificationclass{classtwo}{
  one={{\color{red} (Classified!)      }},
  two={{\color{red} (More Classified!) }}}

% Set base classification and clearance  
\documentclassified[class=classone]
\documentclearance [class=classone, level=two]
\documentclearance [class=classtwo, level=one]


% Code listings
\usepackage{xcolor}
\usepackage{listings}
\lstset{language=TeX, 
        basicstyle=\ttfamily,
        %keepspaces=false,
        breaklines=true,
        commentstyle=\color{gray},
        identifierstyle=\color{blue}}

% Meta info
\title{The \lstinline!classified! package}
\author{Andreas Kl�ckner}

% Begin document
\begin{document}
  \maketitle

% Usage
\section{Usage}
Load the package via \lstinline!\usepackage{classified}!. Set up your classification scheme by \lstinline!\classificationclass{<class>}{<level1>=<action1>, <level2>=<action2>, ...}!. Class could be e.g. \lstinline!NATO! or \lstinline!DLR!. Levels have to be declared in descending order of confidentiality. Use the \lstinline!\classifiedremark{<text>}! macro to insert \lstinline!<text>! on the top and bottom of the current page. If you wish a different placement of the remarks, you can use e.g. \lstinline!<level>=\AddToShipoutPictureFG*{\AtPageTop{<remark>}}!. This will place a remark only at the top of the page. Similarly, there are the commands \lstinline!\AtPageBottom!, \lstinline!\AtPageLeft! and \lstinline!\AtPageRight!.

There are some predefined classification schemes: \lstinline!default! with classifaction levels \lstinline!none!, \lstinline!restricted!, \lstinline!confidential!, \lstinline!secret! and \lstinline!topsecret!. And \lstinline!publication! with the levels \lstinline!none!, \lstinline!final!, \lstinline!paper!, \lstinline!submitted! and \lstinline!draft!.

You can define the document to be classified by \lstinline!\documentclassified[<options>]!. \lstinline!<options>! are \lstinline!key=value! pairs for \lstinline!class! and \lstinline!level!. The default class is \lstinline!default!. The default level is the lowest classification level declared for the given class. This command executes the \lstinline!<action>! macros defined for the selected level in each page. If the \lstinline!\documentclassified! command is not issued, the document classification defaults to \lstinline!none!.

Individual text passages can be marked classifed by \lstinline!\classified[<options>]{<text>}!. The options are the same as for the \lstinline!\documentclassified! macro. This macro only executes the \lstinline!<action>! macro on the current page, if the classification level of \lstinline!<text>! is higher than the document classification.

A clearance level for the document can be set by \lstinline!\documentclearance[<options>]!. The \lstinline!<options>! are again the same as for the previous two commands. This command suppresses the output of the \lstinline!\classfied! commands's \lstinline!<text>! arguments. Only classified text with a level lower or equal to the document clearance will be displayed. Any other text will be replaced by a remark, that text has been suppressed. If no \lstinline!\documentclearance! command is issued, the document clearance defaults to \lstinline!none!.

All the above commands work separately for each classification class. In this way, you can e.g. globally insert \lstinline!publication! remarks and only mark selected sections with \lstinline!default! classification.

% Example
\newpage
\section{Example}
This document has been produced with the following preamble:

\begin{lstlisting}[linewidth=\textwidth]
% Use classified package
\usepackage{xcolor}
\usepackage{classified}

% Define class one classification scheme
\classificationclass{classone}{
  one=\classifiedremark{\colorbox{white}{
                        Level One Classification}}, 
  two=\classifiedremark{\colorbox{red}  {
                        Classification Level Two}}}
                        
% Define class two classification scheme
\classificationclass{classtwo}{
  one={{\color{red} (Classified!)      }},
  two={{\color{red} (More Classified!) }}}

% Set base classification and clearance  
\documentclassified[class=classone]
\documentclearance [class=classone, level=two]
\documentclearance [class=classtwo, level=one]
\end{lstlisting}

If you want to additionally mark selected <text> passages, you can use something like the following:

\begin{lstlisting}[linewidth=\textwidth]
\begin{itemize}
  \item ClassOne-LevelTwo:
        \classified[class=classone, level=two]{<text>}
  \item ClassTwo-LevelOne:
        \classified[class=classtwo, level=one]{<text>}
  \item ClassTwo-LevelTwo:
        \classified[class=classtwo, level=two]{<text>}
\end{itemize}
\end{lstlisting}

This yields the following result. Note the changed markings for ClassOne-LevelTwo on the page borders. Also note the marked ClassTwo-LevelOne <text> and the removed <text> for ClassTwo-LevelTwo.
	
\begin{itemize}
  \item ClassOne-LevelTwo: \classified[class=classone, level=two]{<text>}
  \item ClassTwo-LevelOne: \classified[class=classtwo, level=one]{<text>}
  \item ClassTwo-LevelTwo: \classified[class=classtwo, level=two]{<text>}
\end{itemize}
	
% List of versions
\newpage
\section{List of versions}
\listofversions

\end{document}