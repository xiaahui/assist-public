\documentclass{article}

% Code listings
\usepackage{xcolor}
\usepackage{listings}
\lstset{language=TeX,
        basicstyle=\ttfamily,
        %keepspaces=false,
        breaklines=false,
        commentstyle=\color{gray},
        identifierstyle=\color{blue}}
\usepackage{graphicx}
\usepackage{monochrome}

% Meta info
\title{The monochrome package}
\author{Andreas Kl\"ockner}

% Begin document
\begin{document}
  \maketitle

% Usage
\section{Usage}
Simply load the package via \lstinline!\usepackage{monochrome}!. It will then initialize the \lstinline!xcolor! package with its gray option and add some preprocessing to the \lstinline!\includegraphics! command of the \lstinline!graphicx! package. The package might still need adaptation, as this is only a beta version of the package!

\section{Settings paths}
This package relies heavily on external software. The software is called using the \lstinline!external-base! package. For each external software, the paths need to be known to the package. If the program is on the path, it is found automatically and no change is needed. If not so, please use one of the following commands to set it appropriately:
\begin{itemize}
	\item \lstinline!\PathToGS{path/to/gs.exe}! to set the ghostscript path.\\
	      GhostScript should be provided with your \LaTeX~installation.
	\item \lstinline!\PathToConvert{path/to/convert.exe}! for ImageMagick's \lstinline!convert!.\\
	      ImageMagick is available online: http://www.imagemagick.org/. You will only need the \lstinline!convert! executable.
\end{itemize}

You can also use the configuration file \lstinline!external-config.conf! to store all your settings. The package uses the \lstinline!external-config! package for this funtionality. Please refer to its documentation for more information. Adjust the following example to get started:\\\\
\lstinline!\set@config{monochrome}{\PathToGS{path/to/gs.exe}}!\\
\lstinline!\set@config{monochrome}{\PathToConvert{path/to/convert.exe}}!


\section{Example}
This image is originally colorful: \includegraphics{monochrome_smiley.png}

\end{document}