% Document class
\documentclass{scrartcl}

% Load packages
\usepackage[latin1]{inputenc}
\usepackage[T1]{fontenc}
\usepackage{microtype}
\usepackage[command]{frutiger}
\usepackage[ngerman, english]{babel}
\usepackage{booktabs}
\usepackage[ pdfauthor  = {Andreas Knoblach},
						 pdftitle   = {Frutiger LaTeX Package},
						 colorlinks = true, 
						 linkcolor  = black,  
						 citecolor  = black,
						 urlcolor   = blue
					]{hyperref}

% Define title and author
\title{The \emph{Frutiger} Package}
\subtitle{Frutiger Fonts for \LaTeX2e{}}
\author{Andreas Knoblach}

% Define Blind text
\newcommand{\blindtext}{
	Die hei�e Zypernsonne qu�lte Max und Victoria ja b�se auf dem Weg bis zur K�ste. 1234567890}

% Begin document
\begin{document}

\maketitle
\tableofcontents
\clearpage

%%%%%%%%%%%%%%%%%%%%%%%%%%%%%%%%%%%%
% The Frutiger Fonts               %
%%%%%%%%%%%%%%%%%%%%%%%%%%%%%%%%%%%%
\section{The Frutiger Fonts}
Frutiger is a sans-serif typeface which was created by the Swiss Adrian Frutiger. The available Frutiger fonts within this package are compiled in table \ref{tab:FrutigerFonts}.
\begin{table}[hb]
	\centering
	\caption{Available Frutiger fonts}\label{tab:FrutigerFonts}
	\begin{tabular}{llp{10cm}}
			\toprule
			\textbf{Shape}   & \textbf{Series} & \textbf{Example} \\\midrule
			Upright          & Medium          & \frutiger{\upshape\mdseries\blindtext} \\
			Upright          & Bold            & \frutiger{\upshape\bfseries\blindtext} \\
			Italic          & Medium          & \frutiger{\itshape\mdseries\blindtext} \\
			Italic          & Bold            & \frutiger{\itshape\bfseries\blindtext} \\
			Slanted\footnotemark
					             & Medium          & \frutiger{\slshape\mdseries\blindtext} \\
			Slanted\footref{footnote:italic}
					             & Bold            & \frutiger{\slshape\bfseries\blindtext} \\
			Small caps       & Medium          & \frutiger{\scshape\mdseries\blindtext} \\
			Small caps       & Bold            & \frutiger{\scshape\bfseries\blindtext} \\\bottomrule
	\end{tabular}
\end{table}
\footnotetext{\label{footnote:italic}Actually, there is no slanted shape of Frutiger available. Instead an italic shape is used.}


%%%%%%%%%%%%%%%%%%%%%%%%%%%%%%%%%%%%
% Package Usage                    %
%%%%%%%%%%%%%%%%%%%%%%%%%%%%%%%%%%%%
\section{Package Usage}
The Frutiger package can be easily used within \LaTeX{}\footnote{Note: The fonts are display wrong in the dvi-file but right in the ps- and pdf-file.} and pdf-\LaTeX{} by using the \texttt{\textbackslash usepackage}-command. There are three possibilities to load the Frutiger package

\subsection{Changing \texttt{\textbackslash sffamily}}
In order to redefine \texttt{\textbackslash sffamily} load the Frutiger package without any options, i.\,e. by \texttt{\textbackslash usepackage\{frutiger\}}. The Frutiger font can now be easily activated using \texttt{\textbackslash sffamily}-command. Since this way of loading the Frutiger package effects every usage of \texttt{\textbackslash sffamily} a consistent layout is produced.


\subsection{Providing a \texttt{\textbackslash frutiger\{\}} Command}
In the case that Frutiger Fonts is to be only used in some few cases, the Frutiger package can be loaded by \texttt{\textbackslash usepackage[command]\{frutiger\}}. Herewith only a \texttt{\textbackslash frutiger\{\}} command is defined and no other changes are performed.

\subsection{Changing \texttt{\textbackslash familydefault}}
In order to redefine \texttt{\textbackslash familydefault} it is possible to load the Frutiger package with \texttt{\textbackslash usepackage[family]\{frutiger\}}. In that case, \texttt{\textbackslash sffamily} is redefined to Frutiger and \texttt{\textbackslash familydefault} is redefined to \texttt{\textbackslash sffamily}.

If you use the option \texttt{family} option, it is suggest to select a suitable non-serife math font, e.g. by \texttt{\textbackslash usepackage\{cmbright\}}. It is important that you load the \texttt{cmbright} package before the \texttt{frutiger} package.

%%%%%%%%%%%%%%%%%%%%%%%%%%%%%%%%%%%%
% Package Installation             %
%%%%%%%%%%%%%%%%%%%%%%%%%%%%%%%%%%%%
\section{Package Installation}
The installation procedure for the Frutiger package on different operating systems is explained in the following.

%
% Windows           
%%%%%%%%%%%%%%%%%%%%%%%%%%%%%%%%%%%%
\subsection{Windows: MiKTeX}
In order to correctly install the Frutiger package on a Windows system with MikTeX the following steps have to be performed. 
\begin{enumerate}
	\item Copy the package to any desired path, e.\,g. to \texttt{D:\textbackslash My-LaTeX-Templates\textbackslash Frutiger\textbackslash}. It is important that you do not change the directory structure.
	\item Open the MikTeX Settings GUI and add the Frutiger package root path to the MikTeX root directory list. Click OK.
	\item Update the font map files:
		\begin{enumerate}
			\item Open a DOS console window.
			\item Run \texttt{initexmf -{}-edit-config-file updmap} to edit the font config file.
			\item Add the following lines:\\
						\texttt{Map frutiger-pdf.map} \\
						\texttt{Map frutiger-ps.map}
			\item Save and close the configuration file.
			\item Run \texttt{initexmf -{}-mkmaps} to rebuild the font map files.\footnote{On some systems \texttt{initexmf -{}-mkmaps} has to be executed after every restart of windows. In order to avoid this, you can add a suitable batch file to the Autostart folder in the Start Menu.} 
		\end{enumerate}
\end{enumerate}

%
% Linux           
%%%%%%%%%%%%%%%%%%%%%%%%%%%%%%%%%%%%
\subsection{Linux: TeXLive}
For an installation with TeXLive a user specific installation and an installation for all users is possible. Both possibilities are explained in the following subsections.

%
% Installation by an User         
%
\subsubsection{Installation by an User}
\begin{enumerate}		
	\item Unpack the files into the user's \texttt{texmf/} directory. It is important that you do not change the directory structure.
	\item Update the filename database by executing \texttt{mktexlsr}.
	\item If you use Debian, you have to update the config files by running \texttt{update-updmap}.
	\item Final you have to update the font maps. This is be done by calling \texttt{updmap}.
\end{enumerate}

%
% Installation for all Users           
%
\subsubsection{Installation for all Users}
Please note: if yout want to install the Frutiger package for all users, administrator rights are necessary.
\begin{enumerate}		
	\item Unpack the files to \texttt{/usr/local/share/texmf/}.
	\item Update the filename database by executing \texttt{sudo mktexlsr}.
	\item If you use Debian, you have to update the config files by \texttt{sudo update-updmap}.
	\item Final you have to update the font maps by executing \texttt{sudo updmap-sys}.
\end{enumerate}

%%%%%%%%%%%%%%%%%%%%%%%%%%%%%%%%%%%%
% References                       %
%%%%%%%%%%%%%%%%%%%%%%%%%%%%%%%%%%%%
\addsec{References}
Further information can be found at \url{http://www.macrotex.net/addfont.html}


\end{document}



% eof
